\section*{Introduction}

\subsection*{Digital Images}

A \textbf{digital image} is a representation of a two-dimensional
image as a finite $M \times N$ matrix of digital values, called
picture elements or \textbf{pixels}.

\begin{itemize}
  \item Pixel values typically represent gray levels (often 0 - 255),
    colours, heights, opacities etc.
  \item \textit{Digitization} implies that a digital image is an
    \textbf{approximation} of a real scene.
  \item Common image formats: (i) one sample per point (B \& W,
    Grayscale), (ii) three samples per point (RGB), (iii) four
    samples per point (RGBA).
\end{itemize}

\subsection*{Digital Image Processing}

Two major tasks:

\begin{itemize}
  \item Improvement of pictorial information for human
    interpretation. Examples: image enhancement, image restoration.
  \item Proessing of image data for storage, transmission and
    representation for autonomous machine perception. Examples: image
    segmentation, object recognition, image classification.
\end{itemize}

\subsubsection*{Image Processing $\rightarrow$ Computer Vision Continuum:}

\noindent
%─── First box ───
\begin{minipage}[t]{0.3\linewidth}
  \begin{tblr}{
      width   = \linewidth,
      colspec = { X[c] },
      row{1}  = { font=\bfseries },
      hlines, vlines
    }
    Low-Level Process \\
    \parbox[c][12em][b]{\linewidth}{%
      \textbf{Input}:\\ Image\\
      \textbf{Output}:\\ Image
      \vfill
      \textbf{Examples}: Noise removal, image sharpening
    }
  \end{tblr}
\end{minipage}\hfill
\begin{minipage}[t]{0.3\linewidth}
  \begin{tblr}{
      width   = \linewidth,
      colspec = { X[c] },
      row{1}  = { font=\bfseries },
      hlines, vlines
    }
    Mid-Level Process \\
    \parbox[c][12em][b]{\linewidth}{%
      \textbf{Input}:\\ Image\\
      \textbf{Output}:\\ Attributes
      \vfill
      \textbf{Examples}: Object recognition, image segmentation
    }
  \end{tblr}
\end{minipage}\hfill
\begin{minipage}[t]{0.3\linewidth}
  \begin{tblr}{
      width   = \linewidth,
      colspec = { X[c] },
      row{1}  = { font=\bfseries },
      hlines, vlines
    }
    High-Level Process \\
    \parbox[c][12em][b]{\linewidth}{%
      \textbf{Input}:\\ Attributes\\
      \textbf{Output}:\\ Understanding
      \vfill
      \textbf{Examples}: Scene understanding, autonomous navigation
    }
  \end{tblr}
\end{minipage}

\subsubsection*{Key Stages in Digital Image Processing:}

\begin{figure}[H]
  \centering
  \begin{tikzpicture}[
      node distance=0.8cm and 0.6cm,
      every node/.style={
        draw,
        rectangle,
        minimum width=2.61cm,
        text width=2.3cm,
        align=center
      },
      >={Stealth} % arrow tip
    ]
    % Row 1
    \node (n1) {Image Acquisition};
    \node (n2) [right=of n1] {Image Enhancement};
    \node (n3) [right=of n2] {Image Restoration};

    % Row 2
    \node (n4) [below=of n3] {Morphological Processing};
    \node (n5) [left=of n4] {Segmentation};
    \node (n6) [left=of n5] {Object Recognition};

    % Row 3
    \node (n7) [below=of n6] {Representation \& Description};
    \node (n8) [right=of n7] {Image Compression};
    \node (n9) [right=of n8] {Color Image Processing};

    % Sequential arrows
    \draw[->] (n1) -- (n2);
    \draw[->] (n2) -- (n3);
    \draw[->] (n3) -- (n4);
    \draw[->] (n4) -- (n5);
    \draw[->] (n5) -- (n6);
    \draw[->] (n6) -- (n7);
    \draw[->] (n7) -- (n8);
    \draw[->] (n8) -- (n9);
  \end{tikzpicture}
\end{figure}

\subsection*{Formation of Digital Images}

\begin{enumerate}
  \item \textbf{Image Acquisition:} Illuminate a scene, absorb the
    energy reflected by objects in the scene (various examples: (i)
      x-rays of skeletons, (ii) ultrasounds of unborn babies, (iii)
    electro-microscopic images of molecules).
  \item \textbf{Image Sensing:} Energy lands on sensor material
    responsive to that type of energy, generates a voltage.
    Collections of sensors are \textbf{arranged} to capture images.
  \item \textbf{Image Sampling:} Continuous image $\rightarrow$
    finite set of values (\enquote{grid of pixels}). Finer sampling
    $\rightarrow$ more pixels, more detail. \textbf{Determines
    spatial resolution}.
  \item \textbf{Image Quantization:} For each pixel, continuous value
    $\rightarrow$ value from finite set (e.g. whole number in 0 -
    255). \textbf{Determines intensity level resolution}.
\end{enumerate}

\subsection*{Relationships between Pixels}

\subsubsection*{Neighborhood}

For a pixel $p$ at position $(x, y)$:

\begin{itemize}
  \item \textbf{$N_4$:} $(x + 1, y), (x - 1, y), (x, y + 1), (x, y - 1)$.
  \item \textbf{$N_D$:} $(x + 1, y + 1), (x - 1, y + 1), (x + 1, y -
    1), (x - 1, y - 1)$.
  \item \textbf{$N_8$:} $N_4 \cup N_D$.

\end{itemize}

\noindent
\begin{minipage}{0.3\linewidth}
  \begin{tblr}{
      hlines, vlines,
      colspec={c},
      row{1}={font=\bfseries, c, m}
    }
    4-neighbors \\
    \begin{tikzpicture}[scale=0.6]
      \draw[step=1,gray,very thin] (-1.5,-1.5) grid (2.5,2.5);
      \fill[MaterialBlue200] (0,0)   rectangle (1,1);
      \fill[MaterialRed100]  (1,0)   rectangle (2,1);
      \fill[MaterialRed100]  (-1,0)  rectangle (0,1);
      \fill[MaterialRed100]  (0,1)   rectangle (1,2);
      \fill[MaterialRed100]  (0,-1)  rectangle (1,0);
    \end{tikzpicture}
  \end{tblr}
\end{minipage}\hfill
\begin{minipage}{0.3\linewidth}
  \begin{tblr}{
      hlines, vlines,
      colspec={c},
      row{1}={font=\bfseries, c, m}
    }
    D-neighbors \\
    \begin{tikzpicture}[scale=0.6]
      \draw[step=1,gray,very thin] (-1.5,-1.5) grid (2.5,2.5);
      \fill[MaterialBlue200] (0,0)   rectangle (1,1);
      \fill[MaterialRed100]  (1,1)   rectangle (2,2);
      \fill[MaterialRed100]  (-1,1)  rectangle (0,2);
      \fill[MaterialRed100]  (1,-1)  rectangle (2,0);
      \fill[MaterialRed100]  (-1,-1) rectangle (0,0);
    \end{tikzpicture}
  \end{tblr}
\end{minipage}\hfill
\begin{minipage}{0.3\linewidth}
  \begin{tblr}{
      hlines, vlines,
      colspec={c},
      row{1}={font=\bfseries, c, m}
    }
    8-neighbors \\
    \begin{tikzpicture}[scale=0.6]
      \draw[step=1,gray,very thin] (-1.5,-1.5) grid (2.5,2.5);
      \fill[MaterialBlue200] (0,0)   rectangle (1,1);
      \fill[MaterialRed100]  (1,0)   rectangle (2,1);
      \fill[MaterialRed100]  (-1,0)  rectangle (0,1);
      \fill[MaterialRed100]  (0,1)   rectangle (1,2);
      \fill[MaterialRed100]  (0,-1)  rectangle (1,0);
      \fill[MaterialRed100]  (1,1)   rectangle (2,2);
      \fill[MaterialRed100]  (-1,1)  rectangle (0,2);
      \fill[MaterialRed100]  (1,-1)  rectangle (2,0);
      \fill[MaterialRed100]  (-1,-1) rectangle (0,0);
    \end{tikzpicture}
  \end{tblr}
\end{minipage}

\subsubsection*{Adjacency}

For two pixels $p$ and $q$:
\begin{itemize}
  \item \textbf{4-adjacency:} $q \in N_4(p)$
  \item \textbf{8-adjacency:} $q \in N_8(p)$
  \item \textbf{m-adjacency:} (i) $q \in N_4(p)$ \textbf{or} (ii) $q
    \in N_D(p)$ and $N_4(p) \cap N_4(q) \nsubseteq \mathbf{V}$
\end{itemize}

\noindent
\begin{minipage}{0.3\linewidth}
  \begin{tblr}{
      hlines, vlines,
      colspec={c},
      row{1}={font=\bfseries, c, m}
    }
    4-adjacency \\
    \begin{tikzpicture}[scale=0.6]
      % grid and pixel array
      \draw[step=1,gray,very thin] (-0.5,-0.5) grid (3.5,3.5);
      \fill[MaterialBlue200] (1,2) rectangle ++(1,1);
      \fill[MaterialBlue200] (2,2) rectangle ++(1,1);
      \fill[MaterialBlue200] (1,1) rectangle ++(1,1);
      \fill[MaterialBlue200] (2,0) rectangle ++(1,1);
      % 4-adjacency connections
      \draw[MaterialRed600,dashed,very thick] (1.5,2.5) -- (2.5,2.5);
      \draw[MaterialRed600,dashed,very thick] (1.5,2.5) -- (1.5,1.5);
    \end{tikzpicture}
  \end{tblr}
\end{minipage}\hfill
\begin{minipage}{0.3\linewidth}
  \begin{tblr}{
      hlines, vlines,
      colspec={c},
      row{1}={font=\bfseries, c, m}
    }
    8-adjacency \\
    \begin{tikzpicture}[scale=0.6]
      % grid and pixel array
      \draw[step=1,gray,very thin] (-0.5,-0.5) grid (3.5,3.5);
      \fill[MaterialBlue200] (1,2) rectangle ++(1,1);
      \fill[MaterialBlue200] (2,2) rectangle ++(1,1);
      \fill[MaterialBlue200] (1,1) rectangle ++(1,1);
      \fill[MaterialBlue200] (2,0) rectangle ++(1,1);
      % 8-adjacency connections
      \draw[MaterialRed600,dashed,very thick] (1.5,2.5) -- (2.5,2.5);
      \draw[MaterialRed600,dashed,very thick] (1.5,2.5) -- (1.5,1.5);
      \draw[MaterialRed600,dashed,very thick] (1.5,1.5) -- (2.5,2.5);
      \draw[MaterialRed600,dashed,very thick] (1.5,1.5) -- (2.5,0.5);
    \end{tikzpicture}
  \end{tblr}
\end{minipage}\hfill
\begin{minipage}{0.3\linewidth}
  \begin{tblr}{
      hlines, vlines,
      colspec={c},
      row{1}={font=\bfseries, c, m}
    }
    m-adjacency \\
    \begin{tikzpicture}[scale=0.6]
      % grid and pixel array
      \draw[step=1,gray,very thin] (-0.5,-0.5) grid (3.5,3.5);
      \fill[MaterialBlue200] (1,2) rectangle ++(1,1);
      \fill[MaterialBlue200] (2,2) rectangle ++(1,1);
      \fill[MaterialBlue200] (1,1) rectangle ++(1,1);
      \fill[MaterialBlue200] (2,0) rectangle ++(1,1);
      % 4-adjacency connections
      \draw[MaterialRed600,dashed,very thick] (1.5,2.5) -- (2.5,2.5);
      \draw[MaterialRed600,dashed,very thick] (1.5,2.5) -- (1.5,1.5);
      % m-adjacency diagonal (only center to bottom-right)
      \draw[MaterialRed600,dashed,very thick] (1.5,1.5) -- (2.5,0.5);
    \end{tikzpicture}
  \end{tblr}
\end{minipage}

\subsubsection*{Connectivity}

\begin{itemize}
  \item Pixels $p, q$ connected in $S$ if there exists a path joining
    them consisting entirely of pixels in $S$.
  \item For any pixel $p \in S$, the set of pixels connected to $p$
    in $S$ is called a \textbf{connected component} of $S$.
  \item $S$ is a \textbf{connected set} iff it has only one connected component.
\end{itemize}

\subsubsection*{Region}

A subset $R$ of pixels in an image is a \textbf{region} if $R$ is a
connected set.

\subsubsection*{Boundary}

Set of pixels in a region $R$ that have at leasst one neighbor
outside $R$. Also known as (i) \textbf{border} or (ii) \textbf{contour}.

\subsubsection*{Distance Measures}

Let $p, q, z$ have coordinates $(x, y), (s, t), (v, w)$ respectively.

$D$ is a distance function/metrix if:

\begin{itemize}
  \item $D(p, q) \geq 0$ ($= 0$ iff $p = q$)
  \item $D(p, q) = D(q, p)$
  \item $D(p, z) \leq D(p, q) + D(q, z)$
\end{itemize}

\textbf{Common distance measures:}

\begin{itemize}
  \item \textbf{Euclidean:} $D_e(p, q) = \sqrt{(x - s)^2 + (y - t)^2}$
  \item \textbf{City Block:} $D_4(p, q) = |x - s| + |y - t|$
  \item \textbf{Chessboard:} $D_8(p, q) = \max(|x - s|, |y - t|)$
\end{itemize}

\subsection*{Fuzzy Objects}

\begin{itemize}
  \item A 3-D \textbf{cubic grid} is represented by $\mathcal{Z}^3$
    ($\mathcal{Z} \rightarrow$ set of integers)
  \item \textbf{Object} $\mathcal{O}$ is a \textbf{fuzzy subset}
    $\{p, \mu_0(p) \; | \; p \in \mathcal{Z}^3\}$ ($\mu_0 :
    \mathcal{Z}^3 \rightarrow [0, 1]$)
  \item \textbf{Support} $\theta(\mathcal{O}) = \{p \; | \; p \in
    \mathcal{Z}^3, \mu_0(p) > 0\}$ (set of points with non-zero membership)
  \item \textbf{Background} $\bar{\theta}(\mathcal{O}) =
    \mathcal{Z}^3 - \theta(\mathcal{O})$ (set of points with zero membership)
  \item Images are acquired with finite field of view, so an object
    always has \textbf{bounded support}.
\end{itemize}

\subsection*{Path, Link and Length of Path}

\begin{itemize}
  \item A \textbf{path} $\pi$ in set of points $S$ joining $p$ to $q$
    is a sequence $\langle p = p_1, p_2, \ldots, p_l = q \rangle$
    such that any two successive points $p_i, p_{i+1}$ are adjacent in $S$.
  \item A \textbf{link} is a path of length 2
  \item \textbf{Length of a path} $\pi = \langle p_1, p_2, \ldots,
    p_l \rangle$ in a fuzzy object $\mathcal{O}$ is defined as:
    \begin{equation*}
      \Pi_0(\pi) = \sum_{i=1}^{l-1} \frac{1}{2} \Big(\mu_0(p_i) +
      \mu_0(p_{i+1})\Big) \; || p_i - p_{i+1} ||
    \end{equation*}
\end{itemize}
